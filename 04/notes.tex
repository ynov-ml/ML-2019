\input ../notes-header.tex

\begin{document}

\notetitle{J04}{Recommendation}

Given data about a user, his environment, and some items of in-
terest (training data), determine items to recommend.

We don’t have to find the max $k$.\\
It’s enough to find k within some max $n$.

Examples:
\begin{itemize}
\item Amazon
\item Google New (or Le Monde, etc.)
\item Facebook
\item Twitter
\item Medical testing
\item App store / play store
\item Youtube
\item Advertising
\item Netflix, last.fm, spotify, pandora, \dots
\item Browser (URL recommendations)
\item Search
\end{itemize}

Client value proposition:
\begin{itemize}
\item Find opportunities
\item Reduce choice
\item Explore options
\item Discover long tails
\item Recreation
\end{itemize}

Provider value proposition:
\begin{itemize}
\item Offer unique or additional service (beyond competitors)
\item Customer trust, loyalty
\item Increase sales, CTR, conversions
\item Better understand customers
\end{itemize}

\textbf{Roughly:}

\begin{tabular}{|p{5.7cm}|p{4cm}|}
  \hline
  \topstrut\hbox{Content-based filtering}
  \hbox{\it (filtrage basée sur le contenu)}
  &
    More things similar to what I like\bottomstrut
  \\
  \hline
  \topstrut\hbox{Collaborative filtering}
  \hbox{\it (filtrage collaboratif)}
  &
    More of what other people who like what I like like
    \bottomstrut
  \\
  \hline
  \topstrut\hbox{Knowledge-based filtering}
  \hbox{\it (filtrage basée sur connaissance)}\bottomstrut
  &
    More of what I need.\bottomstrut
  \\
  \hline
\end{tabular}

\textbf{Content-based filtering}

\textit{\purple{More things similar to what I like}}\\
\textit{\red{Plus de ce qui ressemble à ce que j'aime}}

Advantages\vspace{-2mm}
\begin{itemize}
\item [yes!] No need for community
\item [yes!] Possible to compare items 
\end{itemize}

\medskip
Disadvantages\vspace{-2mm}
\begin{itemize}
\item [no] Understand content
\item [yes] Cold start problem
\item [no] Serendipity
\end{itemize}


\textbf{Collaborative filtering}

\textit{\purple{More of what other people who like what I like like}}\\
\textit{\red{Plus de ce que d'autres qui aiment ce que j'aime aiment}}

Advantages

\begin{itemize}
\item [yes!] No need to understand content
\item [yes!] Serendipity
\item [yes!] Learn market
\end{itemize}

Disadvantages

\begin{itemize}
\item [no] User feedback
\item [yes] Cold start problem (users)
\item [yes] Cold start problem (items)
\end{itemize}


\textbf{Knowledge-based filtering}

\textit{\purple{More of what I need}}\\
\textit{\red{Plus de ce qu'il faut}}

Advantages

\begin{itemize}
\item [yes!] Deterministic
\item [yes!] Certainty
\item [no!] Cold start problem
\item [yes!] Market knowledge
\end{itemize}

Disadvantages:

\begin{itemize}
\item [yes] Studies to bootstrap
\item [yes] Static model, doesn't learn from trends
\end{itemize}

\textbf{Utility Matrix}

\begin{itemize}
\item Users (utilisateurs)
\item Items (objets)
\end{itemize}

The goal is to fill in the blanks.

\textbf{Example: books sales at Amazon.}

\begin{tabular}{l|ccccc}
  & $I_1$ & $I_2$ & $I_3$ & $I_4$ & $I_5$ \\
  \hline
  $U_1$ & 1 & & & & \\
  $U_2$ & & & 1 & 1 & 1 \\
  $U_3$ & & 1 & & 1 & 1
\end{tabular}

\textbf{Example: film advice at Netflix.}

\begin{tabular}{l|ccccc}
  & $I_1$ & $I_2$ & $I_3$ & $I_4$ & $I_5$ \\
  \hline
  $U_1$ & 3 & & & & \\
  $U_2$ & & & 5 & 1 & 4 \\
  $U_3$ & & 2 & & 5 & 1
\end{tabular}

{\it But thousands or millions of columns and rows.}

How do we make the matrix?
\begin{itemize}
\item Ask users
\item Observe users
\end{itemize}

That's usually expensive\dots

\textbf{Examples: you are the users, what are the objects}
\begin{itemize}
\item Films
\item Books
\item News
\item Images
\end{itemize}

\textbf{For tomorrow:}
\begin{itemize}
\item Create a git repo, ideally identiable as you.  Send a PR to add it to GIT.md.
\item Add your MNIST code to your repo, in a directory called something recognisable.
\item If GDPR concerns, tell me.  But your career\dots
\end{itemize}

\end{document}
