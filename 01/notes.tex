\input ../notes-header.tex

\begin{document}

\notetitle{J01}{Introduction}

Why ML?
\begin{itemize}
\item Playing with blocks vs doing maths
\item Sometimes we know how to solve problems (e.g., sorting)
\item Sometimes we don't (e.g., recognise a cat, read handwriting on envelopes)
\item Not magic
\end{itemize}

What is ML?
\begin{enumerate}
\item Some algorithms we know how to write
  \begin{enumerate}
  \item Sort numbers
  \item Fly a plane
  \end{enumerate}
\item Some algorithms we don't know how to write (example: drive a car)
  \begin{enumerate}
  \item Drive a car 
  \item Read addresses on envelopes
  \item Detect spam
  \end{enumerate}
\item Maybe we can write programs to write programs when we can't
\item Some terms we used to use for ML
  \begin{itemize}
  \item Artificial intelligence
  \item Expert systems
  \end{itemize}
\end{enumerate}

Disclaimers
\begin{itemize}
\item The literature is overwhelmingly in English
\item Time is short
\item You should plan to spend three hours working on your own per hour in class
\end{itemize}

Types of ML
\begin{enumerate}
\item \fbox{Supervised}
  \begin{enumerate}
  \item Training data: input and correct responses
  \item Regression (continuous) (example: home prices)
  \item Classification (discrete) (example: medical outcome (alive/dead))
  \end{enumerate}
\item \fbox{Unsupervised}
  \begin{enumerate}
  \item Clustering
  \item Deep neural networks
  \item Associative (example: human experience, e.g. from a career)
  \item Dimensionality reduction
  \end{enumerate}
\item \fbox{Reinforcement}
  \begin{enumerate}
  \item Make a choice, get feedback
  \item Online
  \item Can be stochastic (example: predicting weather from local clues)
  \end{enumerate}
\end{enumerate}

Talk about course structure
\begin{itemize}
\item In class: mostly theory, some code, some maths
\item Group work (in class, without professor)
\item Between classes: coding assignments (python)
\item Communication: email, github (ideally use similar names)
\item Help each other via email, github issues, etc.
\item Participative evaluation
\item Don't copy.  Learn.
\item Final project
\end{itemize}

\fbox{Curse of Dimensionality}
\begin{enumerate}
\item \textit{Fléau (ou : malédiction) de la dimension}
\item Volume of unit cube $\pm\epsilon$
\item Distance from $(0,0,\ldots,0)$ to $(1,1,\ldots,1)$
\item Physics: $1/r^{d-1}$
\item It's easy to get lost\dots
\item Richard Ernest Bellman, Dynamic programming, Princeton
  University Press, 1957.
\end{enumerate}

Probability
\begin{enumerate}
\item Event
\item Complement of an event
\item Disjoint (mutually exclusive)
\item Independent events --- knowing one outcome gives no information about other
\item Marginal probability
\item Joint probability
\end{enumerate}

Addition rule: independent events
\begin{displaymath}
  \Pr(A\cup B) = \Pr(A) + \Pr(B)
\end{displaymath}

Addition rule: dependent events
\begin{displaymath}
  \Pr(A\cup B) = \Pr(A) + \Pr(B) - \Pr(A\cap B)
\end{displaymath}

Multiplication rule: independent events
\begin{displaymath}
  \Pr(A\cap B) = \Pr(A) \Pr(B)
\end{displaymath}

Multiplication rule: dependent events
\begin{displaymath}
  \Pr(A\cap B) = \Pr(A\mid B) \Pr(B)
\end{displaymath}

Conditional probability
\begin{displaymath}
  \Pr(A\mid B) = \frac{\Pr(A\cap B)}{\Pr(B)}
\end{displaymath}

\vbox{
\begin{displaymath}
  \cup_i A_i = A \quad\land\quad A_i \cap A_j = \emptyset \implies
\end{displaymath}

\begin{displaymath}
  P(A_1 \mid B) = \frac{\Pr(B\mid A_1) \Pr(A_1)}{\sum_i \Pr(B\mid
      A_1)\Pr(A_1) + \dots + \Pr(B\mid A_k)\Pr(A_k)}
\end{displaymath}
}

\textbf{Statistics}
\begin{enumerate}
\item Goal for a bit: think like a statistician
\item Said differently: goal is to compare reality to a model
\item Or to find a model and then compare.
\item Good statistical models are often relatively simple.
\end{enumerate}

\textbf{What is statistics?}
\begin{enumerate}
\item Identify a question or problem.
\item Collect relevant data on the topic.
\item Analyze the data.
\item Form a conclusion.
\end{enumerate}
Sadly, sometimes people forget 1.
  
Statistics is about making 2--4 efficient, rigorous, and meaningful.

\textbf{What is data science?}
\begin{enumerate}
\item Define the question of interest
\item Get the data
\item Clean the data
\item Explore the data
\item Fit statistical models
\item Communicate the results
\item Make your analysis reproducible
\end{enumerate}

\begin{itemize}
\item What does the public perceive?
\item What takes the most time?
\item What is most often forgotten?
\end{itemize}

Is this the same as what statistics is?

\textbf{Study design}
\begin{enumerate}
\item Anecdote\\[2mm]
  \vbox{
  Some properties of anecdote:
  \begin{itemize}
  \item is data
  \item haphazardly collected
  \item is generally not representative
  \item sometimes result of selective retention
  \item does not accumulate to be representative
  \item might be true (by chance)
  \item is ok to use as hypothesis, but be clear that hypothesis is anecdote
  \end{itemize}
}
\item Study types\\[2mm]
  \vbox{
    \begin{itemize}
    \item Observational
    \item Experimental
    \end{itemize}
  
    What can go wrong?
    \begin{itemize}
    \item Forgetting that association $\ne$ causation
    \item Not random
    \item Confounding variables
    \end{itemize}
  }
\item Observational studies can't conclude causality

\item Observational studies can be
  \begin{itemize}
  \item prospective: identify individuals, collect information
  \item retrospective
  \item we can combine them
  \end{itemize}

\item Experimental studies
  \begin{itemize}
  \item We do stuff
  \item Can conclude causation if properly designed
    \begin{itemize}
    \item controlling: hold other variables constant (e.g., drink pill
      with full glass of water even if we don't care)
    \item randomization: cancel out effects we can't control
    \item replication: enough participants
    \end{itemize}
  \end{itemize}

\item Study types example
  \begin{itemize}
  \item Sunscreen use correlated to skin cancer rates.
  \item Confounding variable
  \end{itemize}

\item Random sampling hazards
  \begin{itemize}
  \item Not actually random
  \item Convenience sample
  \item Non-response bias
  \end{itemize}
\end{enumerate}

\textbf{Variable types}
\begin{itemize}
\item all = numerical + categorical
\item numerical = continuous + discrete
\item categorical = regular + ordinal
\end{itemize}

\textbf{bias vs variance}

Illustrate with bullet holes on a round target.


\textbf{Statistical concepts}

\textbf{Variable types}
\begin{itemize}
\item Input:  Features
\item Input variables measure:  Explanatory variable
\item Output: Response variable
\item Training set
\item Test set (tune parameters) (compare model parameters)
\item Validation set (tune hyperparameters) (measure performance of model)
\item Cross validation
\item Bias - same errors regardless of input (inflexible)
\item Variance - different errors with same input (too flexible)
\end{itemize}

\textbf{Population statistics}
\begin{itemize}
\item \textbf{Deviation} is distance from mean
\item \textbf{Variance} is mean square of deviations
\item \textbf{Standard deviation} is square root of variance
\end{itemize}
\begin{displaymath}
  s^2 = \frac{(\overline x - x_1)^2 + \dotsb (\overline x - x_n)^2}{n-1}
\end{displaymath}
\begin{displaymath}
  \sigma^2 = \frac{(\overline x - x_1)^2 + \dotsb (\overline x - x_n)^2}{n}
\end{displaymath}

\begin{displaymath}
  \mbox{Var}(X) = \sigma^2 = (\overline x - x_1)^2 \Pr(X=x_1) + \dotsb
  (\overline x - x_n)^2 \Pr(X=x_n)
\end{displaymath}

\texttt{Mean}
\begin{itemize}
\item sample mean vs population
\item Sample standard deviation and variance: divide by $n-1$
\item Illustrate with balance beam
\item Illustrate with weights hanging off a balanced beam
\item Illustrate with distribution and balanced on pivot at centre of mass
\end{itemize}

\begin{displaymath}
  \mu = E(X) = \sum w_i x_i = \mathbf{w\cdot x}
\end{displaymath}

\begin{displaymath}
  \mu = E(X) = \sum \Pr(X=x_i) x_i  
\end{displaymath}

\begin{displaymath}
  \mu = E(X) = \int xf(x) \D{x}
\end{displaymath}


\fbox{boxplot-vs-pdf.png}

\begin{enumerate}
\item Distributions
  \begin{itemize}
  \item \fbox{Important: pdf (pmf), cdf, ppf}
    \begin{itemize}
    \item pdf = densité de probabilité
    \item pmf = fonction de masse
    \item cdf = fonction de répartition
    \item ppf = ?
    \end{itemize}
  \item The rest: just so you've heard of them
  \end{itemize}
\item Normal distributions
  \begin{itemize}
  \item Sample mean vs population mean
  \item How close are they?
  \item Point estimate: if you have to guess, this is it
  \item Correction: if I want to be on average weighted right as much possible
  \end{itemize}
\item \fbox{Sampling distributions}
  \begin{itemize}
  \item Sampling mean is unimodal and approximately symmetric
  \item It is centred at population mean.
    \item The standard deviation of the sample mean tells us how far a
      point sample's mean is likely to be from the population mean.
      In other words, how much error we are likely to have in the
      point estimate's mean.  \textbf{Standard error.}
    \item TODO: Generate uniform population, sample,
      and plot sampling distribution
    \item TODO Generate highly skewed population, sample, and plot
      sampling distribution
    \item In real life, we don't have access to the population
      parameters.  We have to \textit{estimate} them from samples.  So
      we can't \textit{know} the standard error (erreur type).
  \end{itemize}
\item \fbox{Confidence intervals}
  \begin{itemize}
  \item Sampling is usually expensive.
  \item Reminder:  Independent random samples!
  \item
    Correct language: ``We are 95\% confident that the population
    parameter is between\dots''
  \item
    Incorrect language: describe the confidence interval as capturing
    the population parameter with a certain probability.
  \item
    This is one of the most common errors: while it might be useful to
    think of it as a probability, the confidence level only quantifies
    how plausible it is that the parameter is in the interval.
  \item
    Another especially important consideration of confidence intervals
    is that they only try to capture the population parameter. Our
    intervals say \textit{nothing} about the confidence of
    \begin{itemize}
    \item capturing individual observations
    \item a proportion of the observations
    \item about capturing point estimates
    \end{itemize}
    Confidence intervals only attempt to capture population
    parameters.
  \end{itemize}
\end{enumerate}

Sample $n$ points, choose an interval around the sample mean.

A 95\% confidence interval means if we sample repeatedly, about 95\%
of the samples will contain the population mean.

\fbox{boxplot illustrations (.png) $\times 2$}

\textbf{Linear Algebra}

$B$ is a basis for $V$ iff any of these conditions are met:
\begin{itemize}
\item $B$ is a minimal generating set of $V$
\item $B$ is a maximal set of linearly independent vectors
\item Every vector $v\in V$ can be expressed in a unique way as a sum of $b_i\in B$
\end{itemize}
The conditions are equivalent.

\textbf{Eigenvectors, eigenvalues}

\begin{displaymath}
  Av = \lambda v
\end{displaymath}

\begin{displaymath}
  Av = \lambda 1 v \;\iff\; (A-\lambda 1)v = 0
\end{displaymath}

\fbox{Eigenvector video}

\end{document}
